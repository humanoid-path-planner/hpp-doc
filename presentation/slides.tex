\section{Introduction}
%
%  Introduction
%
\begin {frame} {Path Planning}
  Given
  \begin{itemize}
  \item A robot (kinematic chain),
  \pause
  \item obstacles,
  \pause
  \item constraints (non-holonomic, manipulation),
  \pause
  \item an initial configuration and
  \item goal configurations,
\end{itemize}
  \pause
  Compute a collision-free path satisfying the constraints from the initial
  configuration to a goal configuration.
\end {frame}

%
%  Historical perspective
%

\begin {frame} {Historical perspective}
  \begin{itemize}
    \item 1998: Move3D,
      \pause
    \item 2001: Creation of Kineo-CAM, transfer of Move3D,
      \pause
    \item 2006: Release of KineoWorks-2, development of HPP based on KineoWorks-2,
      \pause
    \item 2013: kineo-CAM is bought by Siemens,
      \pause
    \item December 2013: development of HPP open-source.
  \end{itemize}
\end {frame}

%
%  Main features
%

\begin {frame} {Main features}
  \begin {itemize}
  \item Numerical constraints at the core of the model
    \begin {itemize}
    \item quasi-static equilibrium
    \item object grasp and placement
    \item explicit and implicit constraints
    \end {itemize}
    \pause
  \item no a priori discretization of paths
    \begin {itemize}
    \item evaluation calls constraint projection
    \item constrained paths need to be checked for continuity (class \texttt{hpp::core::PathProjector})
    \end {itemize}
  \end {itemize}
\end{frame}

\section {Description of the software}

%
%  Overview of the architecture
%

\begin {frame} {Overview of the architecture}
  Modular: collection of packages
  \pause
  \begin{itemize}
    \item installation and dependencies managed by \texttt{cmake} and a \texttt{git}
      submodule: {\tiny\texttt{git://github.com/jrl-umi3218/jrl-cmakemodules.git}},
      \pause
    \item programmed in \texttt{C++},
      \pause
    \item controlled via \texttt{python}
  \end{itemize}
\end {frame}


%
%  Overview of the architecture
%

\begin {frame} {Overview of the architecture}
\parbox {\linewidth} {
  \centerline {
    \includegraphics[width=.6\linewidth]{figures/archi-hpp.png}
  }
}
\end {frame}

%
%  HPP SDK
%

\begin {frame} {Software Development Kit}

Packages implementing the core infrastructure
\begin{itemize}
\item Kinematic chain with geometry
  \begin{itemize}
  \item \texttt{pinocchio}: implementation of kinematic chain with geometry,
    \begin{itemize}
      \item tree of joints (Rotation, Translation, SE3: vector + unit-quaternions),
      \item moving hpp::fcl::CollisionObjects,
      \item forward kinematics,
      \item joint Jacobians,
      \item center of mass and Jacobian,
      \item URDF, SRDF parser.
    \end{itemize}
  \end{itemize}
  \pause
\item Numerical constraints
  \begin{itemize}
  \item \texttt{hpp-constraints}: numerical constraints
    \begin {itemize}
    \item implicit $f (\conf) = (\leq) 0$,
    \item explicit $\conf_{out} = f (\conf_{in})$,
    \item numerical solvers based on Newton-Raphson.
    \end{itemize}
  \end{itemize}
\end{itemize}
\end {frame}

\begin {frame} {Newton-Raphson algorithm}
  \parbox {.5\linewidth} {
    \centerline {
      \includegraphics [width=\linewidth] {figures/seq/romeo-7.png}
    }
  }
  \hspace*{.05\linewidth}
  \parbox {.39\linewidth} {
    Constraints
    \begin {itemize}
    \item quasi-static equilibrium (15)
    \item both hands hold the placard (10)
    \end{itemize}
  }
  \centerline {
    Goal: Generate a configuration satisfying the constraints.
  }
\end {frame}

\begin {frame} {Newton-Raphson algorithm}
  \parbox {.5\linewidth} {
    \centerline {
      \includegraphics [width=\linewidth] {figures/seq/romeo-0.png}
    }
  }
  \hspace*{.05\linewidth}
  \parbox {.39\linewidth} {
    Constraints
    \begin {itemize}
    \item quasi-static equilibrium (15)
    \item both hands hold the placard (10)
    \end{itemize}
  }
  \centerline {
    Shoot random configuration
  }
\end {frame}

\begin {frame} {Newton-Raphson algorithm}
  \parbox {.5\linewidth} {
    \centerline {
      \includegraphics [width=\linewidth] {figures/seq/romeo-1.png}
    }
  }
  \hspace*{.05\linewidth}
  \parbox {.39\linewidth} {
    Constraints
    \begin {itemize}
    \item quasi-static equilibrium (15)
    \item both hands hold the placard (10)
    \end{itemize}
  }
  \centerline {
    Solve linearized system
  }
\end {frame}

\begin {frame} {Newton-Raphson algorithm}
  \parbox {.5\linewidth} {
    \centerline {
      \includegraphics [width=\linewidth] {figures/seq/romeo-2.png}
    }
  }
  \hspace*{.05\linewidth}
  \parbox {.39\linewidth} {
    Constraints
    \begin {itemize}
    \item quasi-static equilibrium (15)
    \item both hands hold the placard (10)
    \end{itemize}
  }
  \centerline {
    Solve linearized system
  }
\end {frame}

\begin {frame} {Newton-Raphson algorithm}
  \parbox {.5\linewidth} {
    \centerline {
      \includegraphics [width=\linewidth] {figures/seq/romeo-3.png}
    }
  }
  \hspace*{.05\linewidth}
  \parbox {.39\linewidth} {
    Constraints
    \begin {itemize}
    \item quasi-static equilibrium (15)
    \item both hands hold the placard (10)
    \end{itemize}
  }
  \centerline {
    Solve linearized system
  }
\end {frame}

\begin {frame} {Newton-Raphson algorithm}
  \parbox {.5\linewidth} {
    \centerline {
      \includegraphics [width=\linewidth] {figures/seq/romeo-4.png}
    }
  }
  \hspace*{.05\linewidth}
  \parbox {.39\linewidth} {
    Constraints
    \begin {itemize}
    \item quasi-static equilibrium (15)
    \item both hands hold the placard (10)
    \end{itemize}
  }
  \centerline {
    Solve linearized system
  }
\end {frame}

\begin {frame} {Newton-Raphson algorithm}
  \parbox {.5\linewidth} {
    \centerline {
      \includegraphics [width=\linewidth] {figures/seq/romeo-5.png}
    }
  }
  \hspace*{.05\linewidth}
  \parbox {.39\linewidth} {
    Constraints
    \begin {itemize}
    \item quasi-static equilibrium (15)
    \item both hands hold the placard (10)
    \end{itemize}
  }
  \centerline {
    Solve linearized system
  }
\end {frame}

\begin {frame} {Newton-Raphson algorithm}
  \parbox {.5\linewidth} {
    \centerline {
      \includegraphics [width=\linewidth] {figures/seq/romeo-6.png}
    }
  }
  \hspace*{.05\linewidth}
  \parbox {.39\linewidth} {
    Constraints
    \begin {itemize}
    \item quasi-static equilibrium (15)
    \item both hands hold the placard (10)
    \end{itemize}
  }
  \centerline {
    Solve linearized system
  }
\end {frame}

\begin {frame} {Newton-Raphson algorithm}
  \parbox {.5\linewidth} {
    \centerline {
      \includegraphics [width=\linewidth] {figures/seq/romeo-7.png}
    }
  }
  \hspace*{.05\linewidth}
  \parbox {.39\linewidth} {
    Constraints
    \begin {itemize}
    \item quasi-static equilibrium (15)
    \item both hands hold the placard (10)
    \end{itemize}
  }
  \centerline {
    Result: a configuration that satisfies the constraints (up to given threshold).
  }
\end {frame}

\begin {frame} {Software Development Kit}

Packages implementing the core infrastructure
\begin{itemize}
\item Path planning
  \begin{itemize}
  \item \texttt{hpp-core}: definition of basic classes,
    \begin{itemize}
      \item path planning problem,
      \item path planning solvers (RRT),
      \item path optimizers (random shortcut),
      \item path projector (random shortcut),
      \item path validation (discretized and continuous),
      \item steering methods (straight interpolation)
    \end{itemize}
  \end{itemize}
\end{itemize}
\end {frame}


\begin {frame} {Extensions}

Packages implementing other algorithms via plugins in \texttt{hpp-corbaserver}
\begin{itemize}
  \item \texttt{hpp-manipulation}: manipulation planning (see next section),
    \pause
  \item any extension for your application.
\end{itemize}
\end {frame}

%
%  Python control
%

\begin {frame} {Python control}

\texttt{hpp-corbaserver}: python scripting through CORBA
\begin{itemize}
\item embed \texttt{hpp-core} into a CORBA server and expose services through 3 \texttt{idl} interfaces:
  \begin{itemize}
  \item \texttt{Robot} load and initializes robot,
  \item \texttt{Obstacle} load and build obstacles,
  \item \texttt{Problem} define and solve problem.
  \end{itemize}
\pause
\item Implement python classes to help user call CORBA services
  \begin{itemize}
    \item \texttt {Robot} automatize robot loading,
    \item \texttt {ProblemSolver} definition problem helper.
  \end{itemize}
\end{itemize}
\end {frame}

%
%  Python control
%

\begin {frame} {Python control}
  Extensions through plugins in \texttt{hpp-corbaserver}
  \begin{itemize}
    \item \texttt {hpp-manipulation-corba:} control of manipulation planning specific classes and algorithms.
  \end{itemize}
\end {frame}

%
%  Visualization through ROS/rviz
%

\begin {frame} {Visualization through gepetto-gui}
  \begin{center}
    \includegraphics [width=.9\linewidth] {figures/gepetto-gui.png}
  \end{center}
  Implemented by package \texttt {hpp-gepetto-viewer}.
\end {frame}

\section {Manipulation planning}

%
%  Manipulation
%

\begin {frame} {Manipulation}
  Class of problem containing:
  \begin{itemize}
    \item A robot: actuated DOFs
    \item Objects: unactuated DOFs
  \end{itemize}
  \pause
  A solution will be a succession of motion of two types:
  \begin{itemize}
    \item The robot moves without constraints. Objects do not move.
    \item The robot moves while grasping the object.
  \end{itemize}
\end {frame}

\begin {frame} {Manipulation}
  \only<1>{2 states:}
  \only<2->{4 transitions:}
  \begin{figure}
    \centering
    \begin{tikzpicture}[>=stealth',auto,node distance=3cm,
      thick,main node/.style={circle,draw,text width=1.7cm,align=center,font=\footnotesize}]
      \setbeamercovered{transparent}
      \node[main node] (nh) {Not holding};
      {\visible<2->{\uncover<2,4->{\path[->] (nh) edge[loop left] node[left, text width=1.2cm, align=right] {Object fixed} (nh);}}}

      \node[main node] (h) [right of=nh] {Holding};
      {\visible<2->{\uncover<2,4->{\path[<-] (h) edge[bend right=45] node[above] {Grasp} (nh);}}}
      {\visible<2->{\uncover<3->{\path[->] (h) edge[bend left=45] node[below] {Ungrasp} (nh);}}}
      {\visible<2->{\uncover<3->{\path[->] (h) edge[loop right] node[right, text width=1.7cm] {Keep the grasp} (h);}}}
    \end{tikzpicture}
    \label{fig:graphEEAndObject}
  \end{figure}
\end {frame}

\begin{frame}{Constraint}
  \uncover<1->{
    \begin{block}{Definition}
      A function $f \in D^1(\mathcal{C}, \mathbb{R}^m)$.
    \end{block}
  }
  \uncover<2->{
    \begin{exampleblock}{Foliation}
      A leaf of a constraint $f$ is defined by:
      $$ L_{f_0}(f) = \left\{\conf \in \mathcal{C} | f(\conf) = f_0 \right\} $$
    \end{exampleblock}
    where $f_0$ is called the \textit{right hand side} of the constraint.
  }
  \uncover<3->{
    \begin{alertblock}{Projection}
      Using a Newton Descent algorithm:
      $$ \conf_{rand} | f(\conf_{rand}) \ne f_0 \Rightarrow \conf_{proj} | f(\conf_{proj}) = f_0 $$
    \end{alertblock}
  }
\end{frame}

\begin{frame}{Constraint}
  Two types of constraints:
  \begin{block}{Configuration}
    Only one leaf is interesting: $L_{0} (f)$.
  \end{block}
  \begin{block}{Motion}
    A leaf also represents reachability space.
  \end{block}
\end{frame}

\begin{frame}{Foliation}
  In the configuration space:

  \begin{columns}
    \column{0.4\textwidth}
    \includegraphics[width=1\textwidth,height=1\textheight,keepaspectratio]{figures/foliation.png}
    \column{0.6\textwidth}
    \begin{block}{2 constraints on motion}
      \begin{itemize}
        \item $f$: position of the object.
        \item $g$: grasp of the object.
      \end{itemize}
    \end{block}
  \end{columns}
\end{frame}

\begin{frame}{Constraint graph}
  \begin{columns}
    \column{0.5\textwidth}
    \only<1> {
      \includegraphics[width=0.9\textwidth,height=0.9\textheight,keepaspectratio]{figures/foliation_path1.png}
    }
    \only<2-> {
      \includegraphics[width=0.9\textwidth,height=0.9\textheight,keepaspectratio]{figures/foliation_path2.png}
    }
    \column{0.5\textwidth}
    \centering
    \begin{tikzpicture}[>=stealth',auto,node distance=1.5cm,
      thick,main node/.style={circle,draw,text width=0.5cm,align=center,font=\footnotesize}]
      \setbeamercovered{transparent}
      \node[main node] (nh) {$L_f$};
      {\visible<2->{\path[->] (nh) edge[loop left, red] node[left, align=right] {f} (nh);}}

      \node[main node] (h) [right of=nh] {$L_g$};
      {\visible<2->{\path[<-] (h) edge[bend right=45, green] node[above] {f} (nh);}}
      {\visible<2->{\path[->] (h) edge[bend left=45, blue] node[below] {f} (nh);}}
      {\visible<2->{\path[->] (h) edge[loop right, yellow] node[right] {g} (h);}}
    \end{tikzpicture}
  \end{columns}
\end{frame}

\begin{frame}[fragile]{Rapidly exploring Random Tree}
  \begin{columns}
    \column{0.4\textwidth}
    \includegraphics<1>  [width=\textwidth,height=\textheight,keepaspectratio]{figures/seq/foliation-project-seq-1.png}
    \includegraphics<2-3>[width=\textwidth,height=\textheight,keepaspectratio]{figures/seq/foliation-project-seq-2.png}
    \includegraphics<4>  [width=\textwidth,height=\textheight,keepaspectratio]{figures/seq/foliation-project-seq-3.png}
    \includegraphics<5>  [width=\textwidth,height=\textheight,keepaspectratio]{figures/seq/foliation-project-seq-4.png}
    \includegraphics<6>  [width=\textwidth,height=\textheight,keepaspectratio]{figures/seq/foliation-project-seq-5.png}
    \column{0.6\textwidth}
    \setbeamertemplate{itemize item}{}% Remove bullets frp, ote,oze sinote,
    \setlength\leftmargini{0em}
    \begin{itemize}[leftmargin=*]
      \item<1-> $\conf_{rand}$ = shoot\_random\_config()
      \item<2-> $\conf_{near}$ = nearest\_neighbor($\conf_{rand}$, $tree$)
      \item<3-> $f_e$, $f_p$ = select\_next\_state($\conf_{near}$)
      \item<4-> $\conf_{proj}$ = project($\conf_{rand}$, $f_e$)
      \item<5-> $\conf_{new}$ = extend($\conf_{near}$, $\conf_{proj}$, $f_p$)
      \item<6-> $tree$.insert\_node( ($\conf_{near}$, $\conf_{new}$, $f_p$) )
    \end{itemize}
  \end{columns}
\end{frame}

\begin{frame}{hpp-manipulation-corba}
  Provides tools to:
  \begin{itemize}
    \item read URDF files of robots and objects;
      \pause
    \item create grasp contraints between a end-effector (robot) and a handle (object);
      \pause
    \item build the graph of constraints;
  \end{itemize}
\end{frame}

%
%  Installation
%

\begin {frame} {Installation and documentation}
  Everything in
  {\scriptsize
    \href{https://humanoid-path-planner.github.io/hpp-doc}
         {\texttt {https://humanoid-path-planner.github.io/hpp-doc}}
  }
\end {frame}

%
%  Keep informed
%

\begin {frame} {Keep informed}
\begin{itemize}
\item github notifications for issues related to individual packages
\centerline {
  \includegraphics [width=\linewidth] {figures/github-watch.png}
}
\end{itemize}
\end {frame}

%
%  Perspectives
